% Custom commands
\newcommand{\R}{\mathbb{R}}
\renewcommand{\thealgocf}{}

% Crossmarks & checkmarks
\newcommand{\cmark}{\ding{51}}
\newcommand{\xmark}{\ding{55}}

% Colored emphasis
\newcommand{\empy}[1]{{\color{darkorange}\emph{#1}}}
\newcommand{\empr}[1]{{\color{blue}\emph{#1}}}

% A command to fetch an image via curl if it doesn't yet exist
\newcommand{\fetchimage}[2]{%
  \IfFileExists{#1}{}{%
    \immediate\write18{curl -sL "#2" -o "#1"}%
  }%
}

% Robust macro to fetch and convert web image to PNG for LaTeX
% Requires shell-escape, curl, and ImageMagick (convert)
\newcommand{\fetchconvertimageonly}[2]{%
  \IfFileExists{#2}{}{%
    \immediate\write18{curl -sL "#1" -o "temp_image_download"}%
    \immediate\write18{convert temp_image_download "#2"}%
    \immediate\write18{rm -f temp_image_download}%
  }%
}

% Robust macro to fetch and convert web image to PNG for LaTeX
% Requires shell-escape, curl, and ImageMagick (convert)
\newcommand{\fetchconvertimage}[3]{%
  \IfFileExists{#2}{}{%
    \immediate\write18{curl -sL "#1" -o "temp_image_download"}%
    \immediate\write18{convert temp_image_download "#2"}%
    \immediate\write18{rm -f temp_image_download}%
  }%
  \includegraphics[#3]{#2}%
}

% Adaptive macro to convert a GIF into frames and animate it
% Usage: \includeGIF{your.gif}{basename}{width}{fps}
\newcommand{\includeGIF}[4]{%
  % % Frame path prefix
  % \def\gifpath{gif_frames}%
  % \def\frameprefix{\gifpath/#2_}%

  % % If frames don't exist, extract them
  % \IfFileExists{\frameprefix000.png}{}{%
  %   \immediate\write18{mkdir -p \gifpath}%
  %   \immediate\write18{convert "#1" "\frameprefix%03d.png"}%
  %   % Count frames using identify
  %   \immediate\write18{identify -format \%n "#1" > \gifpath/#2_framecount.txt}%
  % }%

  % % Read frame count
  % \newread\framecountfile%
  % \openin\framecountfile=\gifpath/#2_framecount.txt%
  % \read\framecountfile to \totalframes%
  % \closein\framecountfile%

  % % Decrement for zero-based indexing (e.g., 10 frames = 000 to 009)
  % \newcount\lastframe
  % \lastframe=\totalframes
  % \advance\lastframe by -1

  % % Pad last frame (e.g., 9 becomes 009)
  % \edef\lastframepad{\ifnum\lastframe<10 00\the\lastframe\else\ifnum\lastframe<100 0\the\lastframe\else\the\lastframe\fi\fi}%

  % % Animate using detected frame range
  % \animategraphics[loop,controls,width=#3]{#4}{\frameprefix}{000}{\lastframepad}%

  \begingroup
  \def\giffile{#1}%
  \def\basename{#2}%
  \def\widthval{#3}%
  \def\fpsval{#4}%
  \def\gifpath{gif_frames}%
  \def\frameprefix{\gifpath/\basename\_}%

  \IfFileExists{\frameprefix000.png}{}{%
    \immediate\write18{mkdir -p \gifpath}%
    \immediate\write18{convert "\giffile" "\frameprefix%03d.png"}%
    \immediate\write18{identify -format %%n "\giffile" > "\gifpath/\basename\_framecount.txt"}%
  }%

  \newread\framecountfile%
  \openin\framecountfile=\gifpath/\basename\_framecount.txt%
  \read\framecountfile to \totalframes%
  \closein\framecountfile%

  \newcount\lastframe
  \lastframe=\totalframes
  \advance\lastframe by -1

  % Format last frame number as zero-padded (e.g., 9 -> 009)
  \def\lastframepad{\ifnum\lastframe<10 00\the\lastframe\else%
    \ifnum\lastframe<100 0\the\lastframe\else%
    \the\lastframe\fi\fi}%

  \animategraphics[loop,controls,width=\widthval]{\fpsval}{\frameprefix}{000}{\lastframepad}%
  \endgroup
}


% Example box
\newcommand{\examplebox}[2]{
\begin{tcolorbox}[colframe=darkcardinal,colback=boxgray,title=#1]
#2
\end{tcolorbox}}

% Theorem environments
\theoremstyle{remark}
\newtheorem*{remark}{Remark}
\theoremstyle{definition}

% Glossary entries
\newacronym{ML}{ML}{machine learning} 