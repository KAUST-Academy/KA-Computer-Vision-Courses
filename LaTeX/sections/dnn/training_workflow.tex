\begin{frame}{Full Training Workflow}
\begin{itemize}
    \item \textbf{Step 1: Initial Setup}
    \begin{itemize}
        \item Start with a pretrained model and just finetune when possible; it saves time and improves results.
        \item Define an initial architecture without regularization (e.g., dropout) or augmentations.
        \item Set up validation strategy and choose an appropriate evaluation loss and metric.
        \item Train the model to get a \textbf{baseline score}.
    \end{itemize}
\end{itemize}
\end{frame}

\begin{frame}{Full Training Workflow}
\begin{itemize}
    \item \textbf{Step 2: Improvement Process}
    \begin{itemize}
        \item Overfitting is common at the start; use regularization techniques like:
        \begin{itemize}
            \item Dropout, batch normalization,...
        \end{itemize}
        \item Perform error analysis to identify weaknesses and choose appropriate augmentations or preprocessing.
        \item Tune hyperparameters: layers, epochs, learning rate, batch size, etc.
        \item Optionally, use ensembling to boost scores (requires more resources).
    \end{itemize}
    \item \textbf{Key Tip:} Track scores and improvements at every step to measure progress effectively.
\end{itemize}
\end{frame}

\begin{frame}{Full Training Workflow}
\begin{itemize}
    \item \textbf{Step 3: Finalization}
    \begin{itemize}
        \item Save the optimized model for deployment.
        \item Use the model for inference in real-world applications.
    \end{itemize}
\end{itemize}
\end{frame} 