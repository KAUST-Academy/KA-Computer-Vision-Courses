% DUL Definition
\begin{frame}[allowframebreaks]{Unsupervised Learning - Definition}
\begin{itemize}
    \item We have a dataset without labels. Out goal is to learn something interesting about the underlying structure of the data:
    \begin{itemize}
        \item Clusters hidden in the dataset.
        \item Outliers: particularly unusual and/or interesting data points.
        \item Useful signals hidden in the noise, e.g., human speech over a noisy background.
    \end{itemize}
\end{itemize}
\end{frame}

% Components of Unsupervised Learning
\begin{frame}[allowframebreaks]{Components of Unsupervised Learning}
\begin{itemize}
    \item \textbf{Data}: Unlabeled data, e.g., images, text, or sensor readings.
    \item \textbf{Model}: A mathematical representation of the data, e.g., a mixture model or a neural network.
    \item \textbf{Objective function}: A measure of how well the model fits the data, e.g., likelihood or reconstruction error.
    \item \textbf{Optimization algorithm}: An algorithm to minimize the objective function, e.g., gradient descent or expectation-maximization.
    \item \textbf{Evaluation metrics}: Measures to assess the quality of the learned model, e.g., silhouette score or clustering accuracy.
    \item \textbf{Applications}: Use cases for unsupervised learning, e.g., clustering, dimensionality reduction, or anomaly detection.
\end{itemize}
\end{frame}

% Comparison with Supervised Learning
\begin{frame}[fragile]{Supervised vs Unsupervised Learning}
  \scriptsize  % or \footnotesize, \small, even \tiny if absolutely necessary :contentReference[oaicite:0]{index=0}
  \begin{table}[h!]
    \centering
    \begin{tabular}{|p{2cm}|p{3cm}|p{3cm}|}
      \hline
      \textbf{Aspect}        & \textbf{Supervised Learning}                                              & \textbf{Unsupervised Learning}                                                                  \\
      \hline
      \textbf{Objective}     & Learn a function \(f\) from labeled input–output pairs.                  & Discover structure or representations in unlabeled data.                                          \\
      \hline
      \textbf{Evaluation}    & Accuracy, precision/recall on held‑out labels.                           & Clustering validity indices (e.g.\ silhouette), reconstruction error.                             \\
      \hline
      \textbf{Cost}          & Methods range from \(\mathcal O(n)\) to \(\mathcal O(n^3)\) per fit.     & k‑means \(\mathcal O(nkd)\), hierarchical \(\mathcal O(n^2)\), PCA \(\mathcal O(nd^2)\).         \\
      \hline
      \textbf{Labels/Clusters} & Fixed, known set of classes.                                           & Number of clusters unknown; must be chosen or inferred.                                           \\
      \hline
      \textbf{Output}        & Classifier or regressor for new inputs.                                  & Cluster assignments, embeddings, density models, or generative samples.                           \\
      \hline
    \end{tabular}
    \caption{Key differences between Supervised and Unsupervised Learning}
  \end{table}
\end{frame}

